\documentclass[12pt,a4paper]{scrartcl} 
\usepackage[utf8]{inputenc}
\usepackage[english,russian]{babel}
\usepackage{indentfirst}
\usepackage{misccorr}
\usepackage{graphicx}
\usepackage{indentfirst}
\usepackage{amsmath}
\begin{document}
	\begin{titlepage}
		\begin{center}
			\large
			МИНИСТЕРСТВО НАУКИ И ВЫСШЕГО ОБРАЗОВАНИЯ РОССИЙСКОЙ ФЕДЕРАЦИИ
			
			Федеральное государственное бюджетное образовательное учреждение высшего образования
			
			\textbf{АДЫГЕЙСКИЙ ГОСУДАРСТВЕННЫЙ УНИВЕРСИТЕТ}
			\vspace{0.25cm}
			
			Инженерно-физический факультет
			
			Кафедра автоматизированных систем обработки информации и управления
			\vfill

			\vfill
			
			\textsc{Отчет по практике}\\[5mm]
			
			{\LARGE Программаная реализация ранга матрицы. \textit{Вариант 6.}}
			\bigskip
			
			1 курс, группа 1ИВТ1-2
		\end{center}
		\vfill
		
		\newlength{\ML}
		\settowidth{\ML}{«\underline{\hspace{0.7cm}}» \underline{\hspace{2cm}}}
		\hfill\begin{minipage}{0.5\textwidth}
			Выполнила:\\
			\underline{\hspace{\ML}} А.\,Н.~Лисова\\
			«\underline{\hspace{0.7cm}}» \underline{\hspace{2cm}} 2023 г.
		\end{minipage}%
		\bigskip
		
		\hfill\begin{minipage}{0.5\textwidth}
			Руководитель:\\
			\underline{\hspace{\ML}} С.\,В.~Теплоухов\\
			«\underline{\hspace{0.7cm}}» \underline{\hspace{2cm}} 2023 г.
		\end{minipage}%
		\vfill
		
		\begin{center}
			Майкоп, 2023 г.
		\end{center}
	\end{titlepage}
	
% Содержание
\section{Введение}
\label{sec:intro}


\subsection{Формулировка цели}
Целью данной работы является написание программы для нахождение ранга матрицы.

\subsubsection{Теория}
Нахождение ранга матрицы способом элементарных преобразований
(методом Гаусса). Под элементарными преобразованиями матрицы понимаются
следующие операции:
    \begin{enumerate}
        \item умножение на число, отличное от нуля;
        \item прибавление к элементам какой-либо строки или какого-либо столбца;
        \item перемена местами двух строк или столбцов матрицы;
        \item удаление "нулевых" строк, то есть таких, все элементы которых равны нулю;
        \item удаление всех пропорциональных строк, кроме одной.
    \end{enumerate}
    \noindent 
Для любой матрицы A всегда можно прийти к такой матрице B,
вычисление ранга которой не представляет затруднений. Для этого следует
добиться, чтобы матрица B была трапециевидной. Тогда ранг полученной
матрицы будет равен числу строк в ней кроме строк, полностью состоящих из
нулей.

Ступенчатую матрицу называют трапециевидной или трапецеидальной,
если для ведущих элементов a1k1, a2k2, ..., arkr выполнены условия k1=1,
k2=2,..., kr=r, т.е. ведущими являются диагональные элементы. В общем виде
трапециевидную матрицу можно записать так:

\label{sec:picexample}
\begin{figure}[h]
	\centering
	\includegraphics[width=0.4\textwidth]{Снимок1.PNG}
\end{figure}

\section{Ход работы}
\subsection{Код выполненной программы}
\begin{verbatim}
#include <iostream>
#include <vector>
#include <stdlib.h>
#include <Windows.h>
#include <cmath>
using namespace std;

int main()
{
	SetConsoleCP(1251);
	SetConsoleOutputCP(1251);
	int row, col, sum = 0, step = 0, sort1, sort2, rank;
	int i, j, k, p, e;
	double tempmath, eps = 0.00001;
	cout << "Введите количество рядов: ";
	cin >> row;
	cout << "Введите количество столбцов: ";
	cin >> col;
	if (row <= 0 || col <= 0)
	{
		cout << "Ошибка. Неверные параметры матрицы." << endl;
		return 0;
	}
	rank = row;
	vector <vector <double>> matrix;
 
	// ввод матрицы
	for (int i = 0; i < row; i++)
	{
		vector <double> temp;
		for (int j = 0; j < col; j++)
		{
			cout << "Введите значение элемента matrix["
				<< i << "][" << j << "]: ";
			cin >> e;
			temp.push_back(e);
		}
		matrix.push_back(temp);
	}
 
	// вывод матрицы
	cout << endl;
	cout << "\nВот как Вы заполнили Вашу матрицу:\n";
	for (int i = 0; i < row; i++)
	{
		for (int j = 0; j < col; j++)
			cout << "[" << i << "]" << "[" << j
			<< "] == " << matrix[i][j] << "\t";
		cout << endl;
	}
	cout << "\n";

	// приведение матрицы по методу Гаусса
		// проверка условия для диагонали ступеней
	if (col > row - 1)
	{
		for (k = 0; k < row - 1; k++)
		{
			// перенос нулей столбца в низ матрицы путём перемены строк (сортировка методом пузырька)
			j = k;
			for (sort1 = k; sort1 < row; sort1++) {
				for (sort2 = k; sort2 < row - 1; sort2++) {
					if (abs(matrix[sort2][j]) < abs(matrix[sort2 + 1][j]))
					{
						for (j = 0; j < col; j++) swap(matrix[sort2][j], matrix[sort2 + 1][j]);
					}
					j = k;
				}
			}
   
			// преобразования к ступенчатому виду
			for (i = k + 1; i < row; i++)
			{
				j = k;
				tempmath = matrix[i][j] / matrix[i - 1 - step][j];
				if (matrix[i][j] == 0)
				{
					step++;
					continue;
				}
				else
				{
					for (j = k; j < col; j++)
					{
						matrix[i - step - 1][j] = matrix[i - step - 1][j] * tempmath;
						matrix[i][j] = matrix[i][j] - matrix[i - step - 1][j];
					}
					step++;
				}
			}
			step = 0;

			// перенос нулевых строк в самый низ и подсчёт ранга матрицы
			for (p = 0; p < col; p++)
				if (matrix[k + 1][p] == 0) sum++;
			if (sum == col)
			{
				for (p = 0; p < col; p++)
				{
					swap(matrix[row - 1][p], matrix[k + 1][p]);
				}
			}
			sum = 0;
		}
	}
	else
	{
		for (k = 0; k < col; k++)
		{
			// перенос нулей столбца в низ матрицы путём перемены строк (сортировка методом пузырька)
			j = k;
			for (sort1 = k; sort1 < row; sort1++) {
				for (sort2 = k; sort2 < row - 1; sort2++) {
					if (matrix[sort2][j] < matrix[sort2 + 1][j])
					{
						for (j = 0; j < col; j++) swap(matrix[sort2][j], matrix[sort2 + 1][j]);
					}
					j = k;
				}
			}

			// преобразования к ступенчатому виду
			for (i = k + 1; i < row; i++)
			{
				j = k;
				tempmath = matrix[i][j] / matrix[i - 1 - step][j];
				if (matrix[i][j] == 0)
				{
					step++;
					continue;
				}
				else
				{
					for (j = k; j < col; j++)
					{
						matrix[i - step - 1][j] = matrix[i - step - 1][j] * tempmath;
						matrix[i][j] = matrix[i][j] - matrix[i - step - 1][j];
					}
					step++;
				}
			}
			step = 0;

			// перенос нулевых строк в самый низ матрицы
			for (p = 0; p < col; p++)
				if (matrix[k + 1][p] == 0) sum++;
			if (sum == col)
			{
				for (p = 0; p < col; p++)
				{
					swap(matrix[row - 1][p], matrix[k + 1][p]);
				}
			}
			sum = 0;
		}
	}

	// округление чисел близких к нулю
	for (i = 0; i < row; i++)
	{
		for (j = 0; j < col; j++)
		{
			if (abs(matrix[i][j]) < eps) matrix[i][j] = 0;
		}
	}

	// вывод приведенной матрицы
	cout << "\nПриведенная матрица:" << endl;
	for (int i = 0; i < row; i++)
	{
		for (int j = 0; j < col; j++)
			cout << "[" << i << "]" << "[" << j
			<< "] == " << matrix[i][j] << "\t";
		cout << endl;
	}
	cout << "\n";

	// получение ранга матрицы
	for (i = 0; i < row; i++)
	{
		for (j = 0; j < col; j++)
		{
			if (matrix[i][j] == 0) sum++;
		}
		if (sum == col) rank--;
		sum = 0;
	}

	// вывод ранга матрицы
	cout << "\nРанг матрицы: " << rank << endl;
	return 0;
}
\end{verbatim}

\begin{figure}[h]
	\centering
	\includegraphics[width=1\textwidth]{Снимок.PNG}
	\caption{Результат работы}\label{fig:par}
\end{figure}






\end{document}